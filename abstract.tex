\begin{abstract}
Because C and C++ objects cannot be relocated, memory allocators for
those languages cannot compact objects and thus can suffer potentially
catastrophic fragmentation. In a classic result, Robson showed that
such allocators can consume up to $(O \log M/m)$, where $M/m$ is the
ratio of the largest and smallest object request
sizes~\cite{robson1977worst}. We present a counterintuitive result: a
memory management algorithm that can perform compaction without
relocating objects. Our approach \emph{meshes} together objects from
separate pages whose virtual offsets do not overlap, compacting them
physically onto a single page while maintaining their virtual
addresses. Meshing depends on widely-available OS support and a
randomized algorithm that provides provably effective compaction with
high probability. We demonstrate that meshing is practical and
effective by implementing a prototype meshing memory allocator.
\end{abstract}
