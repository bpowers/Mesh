\section{Analysis}
\label{sec:analysis}

Pick $m$ random strings of length $b$ with $n$ ones.
Let $X^s$ be the number of copies of string $s$ and note that $X^s \sim Bin(m,p)$ where $p=1/{b \choose n}$. 
We say a pair of strings $s^1$ and $s^2$ are combinable if $s^1\cdot s^2=0$. A set of strings is combinable if they are pairwise combinable.

\subsection{Combining Pairs of Strings}

Partition the set of all binary strings of length $b$ with $n$ ones into the sets
\[A=\{x\in \{0,1\}^b: |x|=n, x_1=1\}\] and \[
B=\{x\in \{0,1\}^b: |x|=n, x_1=0\} \ .\]


%Consider pairing combinable strings greedily. 
In the case where $n=b/2$, the number of unpaired strings is 

\[U=\sum_{s\in A} |X^s-X^{\bar{s}}|\] 
since $s$ is only combinable with its complement $\bar{s}$ when $n=b/2$.

\begin{theorem}
\[D_m(p)/(2p) \leq E(U) \leq D_m(p)/p \ ,\]
where 
\[D_m(p)=2(1-p)^{N-\rounddown{mp}}p^{1+\rounddown{mp}}(\rounddown{mp}+1){N\choose \rounddown{mp}+1}\]
and note that $\sqrt{mp(1-p)}/\sqrt{2} \leq D_m(p)\leq \sqrt{mp(1-p)}$ on the assumption that $1/m\leq p\leq 1-1/m$.
\end{theorem}
\begin{proof}
Let $X\sim Bin(m,p)$.
By the triangle inequality and linearity of expectation:
\[E[U]=\sum_{s\in A} E[|X^s-X^{\bar{s}}|]\]
\[\leq \sum_{s\in A} E[|X^s-mp|] + E[|X^{\bar{s}}-mp|]\]
\[=1/p \cdot E[|X-mp|]\]

Define the event,
\[A=\{X^s\geq mp\geq X^{\bar{s}} \mbox{ or } X^s\leq mp\leq  X^{\bar{s}}\}\]
By linearity of expectation:
\begin{eqnarray*}
E[U]
& \geq & 
\sum_{s\in A} P(A) 
E[|X^s-X^{\bar{s}}| \big | A] \\
& \geq & 1/2 \cdot 
\sum_{s\in A} E[|X^s-mp|] + E[|X^{\bar{s}}-mp|]\\
& = & 1/(2p)  E[|X-mp|]
\end{eqnarray*}

The result follows since by a theorem of De Moivre, the absolute deviation of the binomial distribution is 
\[E[|X-mp|]=D_m(p)\]
\[=2(1-p)^{m-\rounddown{mp}}p^{1+\rounddown{mp}}(\rounddown{mp}+1){N\choose \rounddown{mp}+1}\] 
The bound on $D_m(p)$ is due to Blyth (1980) and  Berend and  Kontorovich (2013).
\end{proof}

\begin{corollary}
For any $n\leq b/2$, \[\lim_{m\rightarrow \infty} E(U)/m=0\] i.e.,  the fraction of strings that are unpaired tends to zero when $m\rightarrow \infty$.
\end{corollary}
\begin{proof}
By the previous theorem, if $n=b/2$, $E(U)/m \leq D_m(p)/(mp)\leq \frac{\sqrt{(1-p)}}{\sqrt{mp}} \rightarrow 0$ as $m\rightarrow \infty$. Note that the bound also holds for $n\leq b/2$ because decreasing the weight of the strings only increases the probability that a pair of strings is combinable.
\end{proof}

\subsubsection{Random Graph Approach}
Consider the graph where nodes correspond to sampled strings where there
is an edge between two nodes iff the strings are compatible. So each
edge is present with probability $q={b-n \choose n}/{b\choose
  n}$. Note that edges are pair-wise independent but not three-wise
independent (e.g., there are fewer triangles than you would
expect). However, the number of incident edges on a node is
$Bin(m-1,q)$.

Let $M$ be the size of the maximum in the graph. This number is the maximum
number of mutually combinable pairs. Had the edges been fully
independent then $E[M]\geq \frac{m}{2}\frac{c}{c+1}$ if $c=q(m-1)/2$
is constant.

Assuming $mq$ is a small constant less than 1, the following lemma gives a good bound.
\begin{lemma}
\begin{eqnarray*}
& & \frac{m}{2} \cdot  \max \left (2-2(1-q)^{m-1}-(m-1) q, \right . \\
& & \left . (m-1) q\left (1-2q+\frac{{b-2n \choose n}}{{b \choose n}}\right )^{m-2}\right ) \\
&\leq & E(M) \\
& \leq & \frac{m}{2} \cdot  \min(1,(m-1)q) \ .
\end{eqnarray*} 
E.g., for $b=32, n=14$ and $m=5000$ we get $78.5 \leq E(M)\leq 81$. 

Since $E(M)$ is monotonically increasing in $q$, it follows that for $q\geq 1-2^{-1/(m-2)}$ we may deduce that, 
\[0.307 \times \frac{m}{2}=(1-\ln(2))\frac{m}{2} \leq E(M)\]
since $2-2(1-q)^{m-1}-(m-1) q$ is maximized at $q=1-2^{-1/(m-2)}$.
%
%(1-q)^(m-2)=1/2
%(m-2) lg (1-q)=-1
%q=1-2^(-1/(m-2))
%Note that the lower bounds are not increasing in $q$ for all values of $q$ but we can always maximize over smaller values of $q$.
%$(m-1) q (1-2(m-2)q)$ is maximized at $mq\approx 1/4$ and at that point takes the value $1/8$. Hence, as soon as $mq\geq 1/4$ we expect to get a matching whose size is at least an eight as large as the maximum possible matching.

It also follows from the upper bound,
\[P(M\geq 1)\leq {m \choose 2} q \ .\]
E.g., for $b=32, n=14$, we need $m\geq 394$ for $P(M\geq 1)\geq 1/2$.
\end{lemma}

\begin{proof}
The upper bound follows from the fact that $M$ is at most the number
of edges in the graph. Hence, $E(M)\leq {m \choose 2} q$.

Let $I$ be the number of \emph{isolated edges}; an edge $uv$ is
isolated if $uv$ is the only edge in the graph incident on $u$ or
$v$. Let $I_e=1$ if $e$ is isolated, and $I_e=0$ otherwise. Then
\begin{eqnarray*}
& & E(I_{uv}) \\
&=& P(uv\in G)P(uw, vw \not \in G ~ \forall w\in V\setminus \{u,v\}| uv\in G)\\
& =& q\left (1-2q+\frac{{b-2n \choose n}}{{b \choose n}}\right )^{m-2} \ .
\end{eqnarray*}
The first lower bound then follows from $E(I)=\sum_e E(I_e)$.

Another lower bound follows from the following idea. Consider the
graph formed by removing all but one incident edge from any node with
two or more incident edges. The remaining graph is a matching. The
expected size of this matching is at least (because some edge removals
may have been counted twice):

\begin{eqnarray*}
& & {m \choose 2} q-\sum_v E(\max(deg(v)-1,0) \\
&=&
{m \choose 2} q
-m\sum_{d=1}^{m-1} {m-1\choose i} q^i (1-q)^{m-1-i}(i-1)\\
&=& {m \choose 2} q-m[(m-1)q-1+(1-q)^{m-1}]\\
&=& m(1-(1-q)^{m-1})-{m \choose 2} q \ .
\end{eqnarray*}
\end{proof}

Note that the expected number of triangles in such a graph is ${m
  \choose 3} {b-n \choose n}/{b\choose n} {b-2n \choose n}/{b\choose
  n}$. For example, if $b=32, n=10, m=1000$, we expect less than two
triangles. Hence, with such parameters, it only makes sense to look for
pairs of blocks that can be combined rather than sets of three or more
blocks that can be combined together.

%Another bound:
%\[E(M) \geq \frac{m}{2} (m-1)q/2
\subsubsection{Random Walk Approach}

Keep on adding strings up to $m$. If a new string combines with an
existing one, remove both. Let $S_t$ be number of strings at time
$t$. Let $q={b-n \choose n}/{b\choose n}=(1-n/b)(1-n/(b-1))\ldots
(1-n/(b-n+1))\geq (1-n^2/(b-n+1))$. Then,

\[P(S_{t+1}=S_t-1)\geq q\]
\[P(S_{t+1}=S_t+1)\leq 1-q\]

If $q\leq 1/2$ then $E[S_t]=O(\sqrt{t})$ and $q\leq 1/2$ if $b\geq
2n^2+n-1$. As a consequence, if $q\leq 1/2$,  we expect the number of
unpaired strings to go to zero.
%\[P(S_{t+1}=S_t-1)\geq 1-(1-q)^{S_t}\geq 1-e^{-S_t q}\]
%\[P(S_{t+1}=S_t+1)\leq (1-q)^{S_t}\]
%\[
%2(1-p)^{N-\rounddown{Np}}p^{\rounddown{Np}}(\rounddown{Np}+1){N\choose \rounddown{Np}+1}
%\leq 
%\]

\subsection{Combining Multiple Strings}
Note that this is only possible if $n\leq b/3$.

\begin{lemma}
Define a graph $G$ where each string corresponds to a node and there
is an edge between two nodes if the two strings are \emph{not}
combinable. Then, the optimal number of combined strings is the
chromatic number of $G$.
%The number of combined strings 
\end{lemma}

Unfortunately, the edges of $G$ are not independent. Had they been
independent, we would have been able to appeal to existing analysis to
show that the expected value of the chromatic number is
\[
\frac{m}{2\log_{1/q} m}\approx \frac{(1-q)m}{2 \ln m}
\]
where $q$ is the probability of being combinable. 

The dependencies between edges means $(u,v)$ and $(v,w)$ present
implies $(u,w)$ are less likely to be present.

However, the following lemma gives a weaker bound for our case:
\begin{lemma}
It is possible to combine strings such that less than $(1+\epsilon)
(m-1)(1-q)+1$ remain with probability at least $1-me^{-\epsilon^2
  (m-1)(1-q)/3}$. For example, if $b=10, n=2, m=10000$, then the final
number of strings is at most $4156$ with probability at least
$1-3.92030 \times 10^{-13}$ by setting $\epsilon=0.1$.
\end{lemma}

\begin{proof}
The result follows because a greedy algorithm can color a graph in
$\Delta+1$ colors where $\Delta$ is the maximum degree of the graph.
Pick an arbitrary node $v$ and let the degree of $v$ be $D$. Then
$E[D]=(m-1)(1-q)$ and by the Chernoff bound
\[P(D\geq (1+\epsilon) (m-1)(1-q)] \leq e^{-\epsilon^2 (m-1)(1-q)/3}
 \ .\]

Hence by the union bound, the probability that all degrees at less
than $(1+\epsilon) (m-1)(1-q)$ is at most $me^{-\epsilon^2
  (m-1)(1-q)/3}$.

\end{proof}
